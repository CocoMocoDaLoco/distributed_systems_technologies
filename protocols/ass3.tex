\documentclass[a4paper,10pt]{article}

\usepackage[usenames,dvipsnames]{color}
\usepackage{comment}
\usepackage[utf8]{inputenc}
\usepackage{listings}

\definecolor{OliveGreen}{cmyk}{0.64,0,0.95,0.40}
\definecolor{Gray}{gray}{0.5}

\lstset{
    language=Java,
    basicstyle=\ttfamily,
    keywordstyle=\color{OliveGreen},
    commentstyle=\color{Gray},
    captionpos=b,
    breaklines=true,
    breakatwhitespace=false,
    showspaces=false,
    showtabs=false,
    numbers=left,
}

\title{VU Distributed Systems Technologies \\
       SS 2013 \\
       Assignment 3}
\author{Jakob Gruber, 0203440}

\begin{document}

\maketitle

\section{Notes}

\section{Theory}

\subsection{Class loading (1 point)}

\emph{Explain the concept of class loading in Java. What different types of class loaders do exist and how do
they relate to each other? How is a class identified in this process? What are the reasons for developers
to write their own class loaders?}

\vspace{3mm}

\subsection{AOP Fundamentals (2 points)}

\emph{Explain the concept of Aspect Oriented Programming (AOP). Think of typical usage scenarios. What
are aspects, concerns, pointcuts and joinpoints, and how do these concepts relate to each other?
Why is it so important to write minimally matching pointcut definitions?}

\vspace{3mm}

\subsection{Weaving Times in AspectJ (1 point)}

\emph{What happens during weaving in AOP? At what times can weaving happen in AspectJ? Think about
advantages and disadvantages of different weaving times.}

\vspace{3mm}

\subsection{Esper Processing Model (2 point)}

\emph{Study the details of the Esper processing model (available in the online reference of Esper). Describe the
core API elements, and illustrate the main EPL query types based on an exemplary event timeline.}

\vspace{3mm}

\end{document}
