\documentclass[a4paper,10pt]{article}

\usepackage[usenames,dvipsnames]{color}
\usepackage{comment}
\usepackage[utf8]{inputenc}
\usepackage{listings}

\definecolor{OliveGreen}{cmyk}{0.64,0,0.95,0.40}
\definecolor{Gray}{gray}{0.5}

\lstset{
    language=Java,
    basicstyle=\ttfamily,
    keywordstyle=\color{OliveGreen},
    commentstyle=\color{Gray},
    captionpos=b,
    breaklines=true,
    breakatwhitespace=false,
    showspaces=false,
    showtabs=false,
    numbers=left,
}

\title{VU Distributed Systems Technologies \\
       SS 2013 \\
       Assignment 2}
\author{Jakob Gruber, 0203440}

\begin{document}

\maketitle

% TODO: testScenario fails in ws
%       "A SOAP fault of type UnknownGridFault"
%       Complete job management bean (a protocol to ensure validity of schedule)
%       A way to bypass transaction rollback in audit interceptor

\section{Notes}

\subsection{Bean types} \label{sec:beans}

There are two types of enterprise beans: \emph{Session}, which perform a task for
a client or may implement a web service; and \emph{Message-driven}, which acts
as a listener for a messaging type.

In this assignment, we only used \emph{Session} beans. These can be further divided
into:

\emph{Stateful}. This bean type has an internal state represented by its instance variables.
It is not shared between clients, and only exists within the scope of a client
conversation. When the session ends, the bean is removed. Stateful beans may not implement
a web service.

Appropriate, if: 

\begin{itemize}
    \item The bean’s state represents the interaction between the bean and a specific client.
    \item The bean needs to hold information about the client across method invocations.
    \item The bean mediates between the client and the other components of the application, presenting a simplified view to the client.
    \item Behind the scenes, the bean manages the work flow of several enterprise beans.
\end{itemize}

\emph{Stateless}. There is no state associated with this bean across method calls. However,
during a method call, a stateless bean may keep its state in instance variables.
Since all instances of this bean are equivalent, the server may pool instances and
assign them to clients as it sees fit. Stateless beans therefore offer better scalability
than other bean types. 

Appropriate, if: 

\begin{itemize}
    \item The bean’s state has no data for a specific client.
    \item In a single method invocation, the bean performs a generic task for all clients. For example, you might use a stateless session bean to send an email that confirms an online order.
    \item The bean implements a web service.
\end{itemize}

\emph{Singleton}. A singleton bean is instantiated once and kept for the lifetime of the server.
Every client accesses the same instance of the bean. State is maintained between invocations,
but not necessarily between crashes and shutdown.

Appropriate, if: 

\begin{itemize}
    \item State needs to be shared across the application.
    \item A single enterprise bean needs to be accessed by multiple threads concurrently.
    \item The application needs an enterprise bean to perform tasks upon application startup and shutdown.
    \item The bean implements a web service.
\end{itemize}

\subsection{EJB and SOAP Internals}

The relevant incoming and outgoing packet streams are shown in the following figures.
The SOAP traffic is taken from HTTP/XML protocol packets, while the EJB traffic is contained
in GIOP (General Inter-ORB Protocol) packets.

As we can see, both basically contain the same data; but the SOAP communication
is marshalled entirely into XML, while EJB uses its own binary serialization and is therefore
much shorter (but less readable).

\begin{figure}
{\small
\begin{lstlisting}[language=XML]
POST /StatisticsService/service HTTP/1.1
Accept-Encoding: gzip,deflate
Content-Type: text/xml;charset=UTF-8
SOAPAction: "http://localhost:8080/StatisticsService/input"
Content-Length: 620
Host: localhost:8080
Connection: Keep-Alive
User-Agent: Apache-HttpClient/4.1.1 (java 1.5)

<soapenv:Envelope xmlns:soapenv="http://schemas.xmlsoap.org/soap/envelope/"
        xmlns:ses="http://session.ejb.ass2.dst/">
   <soapenv:Header xmlns:wsa="http://www.w3.org/2005/08/addressing">
      <ses:arg1>grid1</ses:arg1>
   <wsa:Action>http://localhost:8080/StatisticsService/input</wsa:Action><wsa:MessageID>uuid:ccb3d231-e8ab-4e12-84a6-2f15b2d5f90c</wsa:MessageID><wsa:To>http://localhost:8080/StatisticsService/service</wsa:To></soapenv:Header>
   <soapenv:Body>
      <ses:getStatisticsForGrid>
         <!--Optional:-->
         <arg0>1</arg0>
      </ses:getStatisticsForGrid>
   </soapenv:Body>
</soapenv:Envelope>
\end{lstlisting}
}
\caption{Outgoing SOAP request traffic}
\end{figure}

\begin{figure}
{\small
\begin{lstlisting}[language=XML]
HTTP/1.1 200 OK
server: grizzly/1.9.50
Content-Type: text/xml;charset=utf-8
Transfer-Encoding: chunked
Date: Tue, 30 Apr 2013 12:32:02 GMT

22d
<?xml version='1.0' encoding='UTF-8'?><S:Envelope xmlns:S="http://schemas.xmlsoap.org/soap/envelope/"><S:Header><Action xmlns="http://www.w3.org/2005/08/addressing">http://localhost:8080/StatisticsService/output</Action><MessageID xmlns="http://www.w3.org/2005/08/addressing">uuid:dd273c77-153a-4c30-b7c4-e94ea1596ae9</MessageID><RelatesTo xmlns="http://www.w3.org/2005/08/addressing">uuid:ccb3d231-e8ab-4e12-84a6-2f15b2d5f90c</RelatesTo><To xmlns="http://www.w3.org/2005/08/addressing">http://www.w3.org/2005/08/addressing/anonymous</To></S:Header><S:Body>
10b
<ns2:getStatisticsForGridResponse xmlns:ns2="http://session.ejb.ass2.dst/"><return>&lt;?xml version="1.0" encoding="UTF-8" standalone="yes"?&gt;&lt;stats&gt;&lt;name&gt;grid1&lt;/name&gt;&lt;/stats&gt;</return></ns2:getStatisticsForGridResponse></S:Body></S:Envelope>
0
\end{lstlisting}
}
\caption{Incoming SOAP response traffic}
\end{figure}

\begin{figure}
{\small
\begin{lstlisting}
GIOP.......p...*...................&...d....S1AS-ORB............RootPOA....
#RFMBase#......a89608981053636609_RBusiness_dst.ass2.ejb.session.interfaces._IJobManagementBean_Remote-EJBObject....... .>Z..c..................Z..k............login.......................................NEO....................#IDL:omg.org/CORBA/WStringValue:1.0.........h.a.n.s.i...................p.w.1
\end{lstlisting}
}
\caption{Outgoing EJB request traffic}
\end{figure}

\begin{figure}
{\small
\begin{lstlisting}
GIOP...........*........NEO................5IDL:dst/ass2/ejb/session/_exception/AssignmentEx:1.0............file:/tmp/dst/gfembed6532460383916994258tmp/applications/ass2-ejb/ file:/tmp/dst/gfembed6532460383916994258tmp/generated/ejb/ass2-ejb......YRMI:dst.ass2.ejb.session.exception.AssignmentException:C541A83F0F5CCDCE:566BE4E93509FD43.......................
...#IDL:omg.org/CORBA/WStringValue:1.0.....:...6...I.n.c.o.r.r.e.c.t. .u.s.e.r. .o.r. .p.a.s.s.w.o.r.d.........
...ERMI:[Ljava.lang.StackTraceElement;:CD38F9930EA8AAEC:6109C59A2636DD85...........).......BRMI:java.lang.StackTraceElement:CD38F9930EA8AAEC:6109C59A2636DD85......................N...d.s.t...a.s.s.2...e.j.b...s.e.s.s.i.o.n...J.o.b.M.a.n.a.g.e.m.e.n.t.B.e.a.n.....................J.o.b.M.a.n.a.g.e.m.e.n.t.B.e.a.n...j.a.v.a.............D.......l.o.g.i.n...............................J...s.u.n...r.e.f.l.e.c.t...N.a.t.i.v.e.M.e.t.h.o.d.A.c.c.e.s.s.o.r.I.m.p.l.................<...N.a.t.i.v.e.M.e.t.h.o.d.A.c.c.e.s.s.o.r.I.m.p.l...j.a.v.a...........p....GIOP...........*...i.n.v.o.k.e.0...............9.......$.......x...........0.......i.n.v.o.k.e.0...............+...............R...s.u.n...r.e.f.l.e.c.t...D.e.l.e.g.a.t.i.n.g.M.e.t.h.o.d.A.c.c.e.s.s.o.r.I.m.p.l94...............D...D.e.l.e.g.a.t.i.n.g.M.e.t.h.o.d.A.c.c.e.s.s.o.r.I.m.p.l...j.a.v.a.......................Y...........0...2...j.a.v.a...l.a.n.g...r.e.f.l.e.c.t...M.e.t.h.o.din...................M.e.t.h.o.d...j.a.v.a.......................................t...o.r.g...g.l.a.s.s.f.i.s.h...e.j.b...s.e.c.u.r.i.t.y...a.p.p.l.i.c.a.t.i.o.n...E.J.B.S.e.c.u.r.i.t.y.M.a.n.a.g.e.r...........(...0...E.J.B.S.e.c.u.r.i.t.y.M.a.n.a.g.e.r...j.a.v.a...................r.u.n.M.e.t.h.o.d...............d.......................p...........d...................
[...]
RRMI:java.util.Collections\U0024UnmodifiableList:0FB2D524A8F804A3:FC0F2531B5EC8E10..l...
...:RMI:java.util.ArrayList:F655154F32815380:7881D21D99C7619D..................
...IRMI:org.omg.custGIOP.......`...*om.java.util.ArrayList:F655154F32815380:7881D21D99C7619D.a.n...................L............
\end{lstlisting}
}
\caption{Incoming EJB response traffic}
\end{figure}

\subsection{Transparent Injection}

Java allows instrumentation using so-called agents, which are deployed as JAR files.
A command-line parameter called \verb|javaagent| points to the JAR file containing the agent, while
a parameter called \verb|Premain-Class| specifies which class to run to initialize the agent.

The \lstinline|premain()| method is executed by the JVM before the \lstinline|main()|
method of the actual program; within \lstinline|premain()|, the agent can hook itself
into the JVM systems as required. In this assignment, we registered an implementation
of the \lstinline|ClassFileTransformer| class, which is called whenever a new class
definition is loaded.

A \lstinline|ClassFileTransformer| can inspect and modify the entire definition of
a class before it is used by the actual program. We use this functionality to
inject a short code snippet into each constructor of each class annotated with \lstinline|@Component|,
which calls into \lstinline|IInjectionController.initialize()| to initialize itself
automatically.

This is accomplished using \verb|javassist|, a library for Java bytecode manipulation.
For each relevant class constructor, \verb|javassist| compiles our short source
code snippet into bytecode, and inserts it at the end of the constructor.

The modified bytecode is then returned by the \lstinline|ClassFileTransformer| and
used by the application.


\section{Theory}

\subsection{EJB Lifecycles (1 Point)}

\emph{
Explain the lifecycle of each bean type defined in the EJB 3.1 specification. What optimizations can the
EJB container perform for the respective type? Also think about typical use cases the respective bean
type provides to the developer.}

\vspace{3mm}

\emph{Stateful.} When the client obtains a reference to a stateful bean, it is first
initialized by the server: dependencies are injected, and any \lstinline|@PostConstruct| methods
are called. Once this completes, the bean is ready to be used.

At any time, the server may decide to passivate the bean; when entering this stage,
\lstinline|@PrePassivate| methods are called, and when exiting it \lstinline|@PostActivate| methods
are called.

Once the client has finished using the bean, a method annotated \lstinline|@Remove| is called.
Upon completion, \lstinline|@PreDestroy| methods are executed and the bean may be garbage collected by the server.

\vspace{3mm}

\emph{Stateless.} A stateless bean is either nonexistent, or ready to be used. Typically, the server
creates a pool of stateless beans. Again, any methods annotated with \lstinline|@PostConstruct| are
called and at the end of its lifecycle, \lstinline|@PreDestroy| methods are executed.

\vspace{3mm}

\emph{Singleton.} The lifecycle of a singleton bean is similar to that of a stateless bean, except
that only exists once instead of several times in a pool. If it is annotated \lstinline|@Startup|,
the bean is constructed as soon as the server starts up.

\vspace{3mm}

\emph{Message-Driven.} Similar to a stateless bean.

\vspace{3mm}

For further elaboration, see section \ref{sec:beans}.

\subsection{Dependency Injection (2 Points)}

\emph{
Explain the way dependency injection is performed by the EJB container. What kind of resources may
be injected into a bean, and what are the different annotations that can be used?}

\vspace{3mm}

% TODO

The following resources can be injected into beans: almost any Java class, session beans, 
Java EE resources, persistence contexts, producer fields, objects returned by producer methods,
web service references, remote beans references.

Usable annotations are:

\begin{itemize}
    \item \lstinline|Resource|: Resources such as data sources
    \item \lstinline|EJB|: Session EJB, Web beans
    \item \lstinline|WebServiceRef|: Web service references
    \item \lstinline|PersistenceContext|: Container-managed \lstinline|EntityManager|
    \item \lstinline|PersistenceUnit|: Container-managed \lstinline|EntityManagerFactory|
\end{itemize}

\subsection{Java Transaction API (2 Points)}

\emph{
The EJB architecture provides a mechanism for distributed transactions. Explain the two ways how
transactions can be defined. How is the concept of distributed transactions accomplished behind the
scenes, i.e. what tasks have to be performed by the EJB container?}

\vspace{3mm}

Transactions may be either container-managed, or user-managed. It is not necessary to explicitly 
state start- and endpoints of container-managed transactions. Instead, transactions start
before the beginning of a method, and end just before it completes. Each method can
be associated with a specific transaction attribute from: \lstinline|Required, RequiresNew, Mandatory, NotSupported, Supports, Never|.
These specify what how to handle method calls from within an existing transaction context.

A container-managed transaction is automatically rolled back on a thrown system exception.
Likewise, a rollback can be initiated by calling \lstinline|EJBContext.setRollbackOnly()|.

Bean-managed (or user-managed) transactions must explicitly mark start- and endpoints.
While this requires more code, it also allows for more flexibility since methods can now
be associated with more than one transaction.

Distributed transactions are handled by passing responsibilities to one or more transaction
managers. Commits are handled using the two-phase commit: first, the manager tells each
transaction branch to verify if it can still commit, and only if a commit is possible
for each branch is the transaction actually committed. Otherwise a rollback instructions
is sent to all branches.

\subsection{Remoting Technologies (1 Point)}

\emph{Compare EJB remoting and Web services. When would you use one technology, and when the other? Is
one of them strictly superior? How do these technologies relate to other remoting technologies that you
might know from other lectures (for instance, Java RMI, CORBA, or even socket programming)?}

\vspace{3mm}

First of all, EJB is obviously related to the JVM and cannot interoperate with other languages
and technologies. Web services on the other hand are completely platform-agnostic.

If performance is an issue, then EJB will be a better choice. Web services have to communicate
through extra layers, and need to marshall/unmarshall all parameters.

Both EJB and web services are well defined. Stable standards exist for both SOAP and EJB.

EJB provides additional functionality such as transaction management, connection pooling, naming and
lookups.

Public vs. private: if the application will only be used internally by (more or less) trusted clients,
EJB is probably a good choice. On the other hand, web services handle publically exposed interfaces better.

Java RMI is somewhat comparable to EJB. It offers less functionality, but otherwise has very similar
advantages and disadvantages as EJB. CORBA is a little more low level, and not tied to any single language.
Socket programming is of course as low as we can get. The lower-level a protocol is, the less services it
offers, and the less overhead it has. On the other hand, programming lower level protocols often requires
manually taking care of details which are handled automatically in higher level protocols. In the end,
there is no single best choice - the decision always has to be made tailored to the project
and to the developers.

\end{document}
