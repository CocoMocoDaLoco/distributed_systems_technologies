\documentclass[a4paper,10pt]{article}
\usepackage{comment}
\usepackage[utf8]{inputenc}

\title{VU Distributed Systems Technologies \\
       SS 2013 \\
       Assignment 1}
\author{Jakob Gruber, 0203440}

\begin{document}

\maketitle

\section{Notes}

\subsection{Choice of inheritance pattern}

\begin{comment}
InheritanceType: JOINED; Common table for base type, additional infos stored 
separately. SINGLE_TABLE sounds nice for our needs, but support is optional.

MappedSuperclass: the superclass contains persisted information, but is not
mapped as an entity by the JPA provider. State & mapping information common to
multiple entity classes. Non-entity superclasses don't support persistence,
and abstract entities are mapped (which we don't need).
\end{comment}

\subsection{Date aggregation in JPQL}

\begin{comment}
JPQL does not support date aggregation in any reasonable form, which means we
can't do summation within the query. Extract all rows individually, then
create the sum within java code.
\end{comment}

\subsection{Indices, pros and cons}

\begin{comment}
When is the User.password index helpful, when could it be harmful. Optional:
large-scale tests.
\end{comment}

\section{Theory}

\subsection{Annotations vs. XML (1 Point)}

\emph{In the previous tasks you already gained some experiences using annotations and XML. What are the
pros and cons of each approach, when would you use which one? Answering these questions, also keep
in mind maintainability and the different roles usually involved in software projects.}

\subsection{6. Versioning (1 Point)}

\emph{JPA provides a feature called versioning. Why and under which circumstances can this feature be useful?
Think about a situation where optimistic locking may result in an (desired) exception.}

\subsection{7. Read-Locks (2 Point)}

\emph{The EntityManager allows the programmer to set Read-locks on specified objects. What are the conse-
quences on concurrent threads when one thread sets such a lock? Think about use-cases this behaviour
may be adequate for. What problems can arise?}

\subsection{8. Database Isolation-Levels (2 Points)}

\emph{Have a look at the different isolation-levels modern databases provide. What kind of problems might
occur due to concurrency issues at what kind of level? Is it really necessary to protect your application
against every type of flaw?}

\subsection{9. Database Indices (2 Point)}

\emph{What is the purpose and functioning of a database index? Using which data structures do database
management systems typically store an index internally, and what are important characteristics of these
data structures? What is the basic tradeoff of using an index, what are its limitations? Think of two
concrete examples - one in which an index leads to an improvement and one in which the index is useless
(i.e., does not lead to an improvement).}

\subsection{10. NoSQL Databases (2 Point)}

\emph{What general types of NoSQL databases exist? Name prominent examples for each type of database,
and argue when you should be using them (and also, when you should specifically not use them).}

\end{document}
